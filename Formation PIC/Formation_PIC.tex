\documentclass[a4paper]{article}
\usepackage[T1]{fontenc}
\usepackage[utf8]{inputenc}
\usepackage{graphicx}
\usepackage[french]{babel}
\usepackage[a4paper]{geometry}
\usepackage{enumerate}
\usepackage{circuitikz}
\usepackage{tikz}
\usetikzlibrary{shapes,arrows,fit}
\usepackage{amsmath}
\usepackage{amssymb}
\usepackage{mathrsfs}
\usepackage{stmaryrd}
\usepackage{float}
\geometry{scale=0.80}

\author{Guillaume Dib}
\title{Formation dsPIC}

\begin{document}

\maketitle
\clearpage

\tableofcontents
\clearpage

%----------------------------------------------Introduction-----------------------------------------------------
\section{Introduction} 
%----------------------------------------------A propos du projet---------------------------------------------
	\subsection{A propos du projet}

Formation PIC pour 7robot.

%----------------------------------------------Documents nécessaires-----------------------------------------
	\subsection{Documents nécessaires}

 \begin{itemize}
 	\item[$\diamond$] Datasheet du dsPIC33FJ64MC80X.
 	\item[$\diamond$] "16bit\_Language\_Tool\_Librairie" pour les fonctions du compilateurs.
\end{itemize}

%----------------------------------------------Logiciels nécessaires-----------------------------------------
	\subsection{Logiciels nécessaires}
	
\begin{itemize}
 	\item[$\diamond$] IDE : MPLABX
 	\item[$\diamond$] Compilateur : xC16
\end{itemize}

%----------------------------------------------Architecture matérielle-----------------------------------------
\section{Architecture matérielle}


\end{document}
